\documentclass{article}
\usepackage[utf8]{inputenc}
\usepackage{xcolor}

\title{Dijkstra e Problemas de Min-Max e Max-Min}
\author{Yan Sabarense}
\date{\today}

\begin{document}

\maketitle

\section{Introdução}

O algoritmo de Dijkstra é um método eficiente para encontrar o caminho mais curto em um grafo com arestas de pesos positivos. Este documento descreve a teoria por trás do algoritmo e suas variações para resolver os problemas de min-max e max-min de gargalos em grafos, com uma descrição das funções principais utilizadas na implementação.

\section{Algoritmo de Dijkstra}

O algoritmo de Dijkstra calcula o caminho mais curto entre um nó inicial e todos os outros nós de um grafo, utilizando uma fila de prioridade para selecionar o próximo nó a ser processado com a menor distância conhecida. 

\subsection{Função \texttt{dijkstra}}

Esta função é responsável por:

\begin{itemize}
    \item Inicializar a distância de todos os nós como infinita, exceto o nó inicial, cuja distância é zero.
    \item Utilizar uma fila de prioridade para expandir os nós de acordo com a menor distância acumulada.
    \item Atualizar as distâncias dos nós adjacentes se um caminho mais curto for encontrado.
\end{itemize}

\section{Problema de Min-Max Gargalo}

Para minimizar o maior peso de uma aresta em um caminho, adaptamos o algoritmo de Dijkstra para priorizar caminhos que minimizem o maior peso entre as arestas.

\subsection{Função \texttt{dijkstra\_min\_max}}

Esta função é uma variação da função original de Dijkstra e:

\begin{itemize}
    \item Calcula o caminho com o menor valor máximo de aresta.
    \item Atualiza os pesos usando a função \texttt{max()} para garantir que o maior peso no caminho seja o menor possível.
    \item Expande os nós de acordo com essa lógica, sempre priorizando o caminho que minimiza o peso máximo.
\end{itemize}

\section{Problema de Max-Min Gargalo}

Para maximizar o menor peso de uma aresta em um caminho, modificamos o algoritmo de Dijkstra para utilizar a função \texttt{min()} ao atualizar o peso de cada caminho.

\subsection{Função \texttt{dijkstra\_max\_min}}

Esta função visa:

\begin{itemize}
    \item Calcular o caminho que maximiza o menor valor de aresta presente.
    \item Utilizar a função \texttt{min()} para preservar a maior capacidade mínima possível entre os caminhos.
    \item Expandir os nós priorizando aqueles que mantêm o maior valor mínimo.
\end{itemize}

\section{Conclusão}

O algoritmo de Dijkstra e suas variações são ferramentas essenciais para resolver problemas de caminhos mínimos e de otimização em grafos. A capacidade de adaptar o algoritmo para cenários de min-max e max-min é particularmente útil para aplicações que envolvem a maximização ou minimização de gargalos em redes de transporte ou comunicação, por exemplo.

\end{document}
