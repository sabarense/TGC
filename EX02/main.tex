\documentclass{article}
\usepackage[utf8]{inputenc}
\usepackage{listings}
\usepackage{xcolor}

\title{Gerador de Subgrafos}
\author{Yan Sabarense}
\date{\today}

\lstset{ 
    language=C++,
    basicstyle=\ttfamily\small,
    keywordstyle=\color{blue},
    commentstyle=\color{gray},
    stringstyle=\color{red},
    numbers=left,
    numberstyle=\tiny\color{gray},
    stepnumber=1,
    numbersep=10pt,
    showspaces=false,
    showstringspaces=false,
    breaklines=true,
    tabsize=4
}

\begin{document}

\maketitle

\section{Introdução}

Este documento descreve o código C++ que gera todos os subgrafos possíveis de um grafo completo com um número dado de vértices. O código também exibe o número total de subgrafos gerados e imprime cada subgrafo.

\section{Descrição do Código}

O código está dividido em duas partes principais: a função \texttt{gerarSubgrafos} e a função \texttt{main}. A seguir, descrevemos o funcionamento de cada uma dessas partes.

\subsection{Função \texttt{gerarSubgrafos}}

A função \texttt{gerarSubgrafos} é responsável por gerar e exibir todos os subgrafos de um grafo completo. Sua assinatura é:

\begin{lstlisting}
void gerarSubgrafos(int numeroVertices);
\end{lstlisting}

\subsubsection*{Parâmetros}

\begin{itemize}
    \item \textbf{numeroVertices} - Um número inteiro que representa a quantidade de vértices no grafo completo.
\end{itemize}

\subsubsection*{Funcionamento}

1. **Cálculo do número de arestas**: O número de arestas de um grafo completo com \( n \) vértices é calculado pela fórmula:
   \[
   \text{totalArestas} = \frac{n(n-1)}{2}
   \]
   Esse cálculo é feito na linha:

\begin{lstlisting}
int totalArestas = (numeroVertices * (numeroVertices - 1)) / 2;
\end{lstlisting}

2. **Cálculo do número total de subgrafos**: O número total de subgrafos de um grafo completo é \( 2^{(\text{número de vértices} + \text{número de arestas})} \). Isso ocorre porque cada vértice e cada aresta pode estar presente ou ausente em um subgrafo. O cálculo é feito da seguinte forma:

\begin{lstlisting}
int totalSubgrafos = 1 << (totalVertices + totalArestas);
\end{lstlisting}

3. **Geração dos subgrafos**: Utilizando um loop, o código gera todos os possíveis subgrafos iterando sobre cada combinação de vértices e arestas. O subgrafo atual é representado por um número binário, onde cada bit indica se um vértice ou aresta está presente no subgrafo.

4. **Impressão dos subgrafos**: A função imprime cada subgrafo, indicando quais vértices e arestas estão presentes.

\subsection{Função \texttt{main}}

A função \texttt{main} serve como ponto de entrada do programa. Ela solicita ao usuário o número de vértices do grafo e chama a função \texttt{gerarSubgrafos} para gerar e exibir os subgrafos.

\begin{lstlisting}
int main() {
    int numeroVertices;
    
    cout << "Informe o número de vértices: ";
    cin >> numeroVertices;

    gerarSubgrafos(numeroVertices);

    return 0;
}
\end{lstlisting}

\subsection{Exemplo de Execução}

Ao executar o programa e informar que o grafo completo tem 3 vértices, o programa exibirá o seguinte resultado:

\begin{verbatim}
Informe o número de vértices: 3
Total de subgrafos: 64
Subgrafo 1: { }
Subgrafo 2: { v1 }
Subgrafo 3: { v2 }
Subgrafo 4: { v3 }
...
\end{verbatim}

\section{Conclusão}

Este código implementa um gerador de subgrafos de um grafo completo em C++. A partir do número de vértices informado pelo usuário, o programa calcula o número total de subgrafos possíveis e imprime cada subgrafo gerado. O código é eficiente para pequenos valores de \( n \), mas o número de subgrafos cresce exponencialmente com o aumento do número de vértices.

\end{document}
